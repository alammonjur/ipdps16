\section{Introduction}

A priority queue is one of popular data structures that is being used for various applications including routing, anomaly prioritization, shortest path search, and scheduling~\cite{web,ah1,ah2,ah3}.
A priority queue is a data structure in which each element has a priority and a dequeue operation removes and returns the highest priority element in the queue. 
It is a basic component for scheduling used in most routers and event driven simulators \cite{hw1,fpga1}.

There are several hardware-based implementations of a priority queue~\cite{hw1,hw2,hw8,hw9,fpga1,fpga2,fpga3} to handle a large volume of elements.
The {\it Systolic Arrays} and {\it Shift Registers} based approaches \cite{hw8,hw9}, for example, are not scalable and require $O(n)$ comparators for $n$ nodes. 
FPGA-based pipelined heap is presented by Ioannou {\it et. al} \cite{fpga1}. 
This architecture is scalable and can run for 64K nodes without compromising performance, but it takes at least 3 clock cycles to complete a single stage. 
The calendar queues \cite{hw1} incur significant hardware cost when they need to support a large priority set.

Moreover, all of exisiting works do not address a {\it hole} generated by parallel {\it min-delete} operations followed by {\it insertion} operations. 
Since holes occupy storage elements but do not have valid data, retaining holes incurs additional overhead.
Holes also lead to an unbalanced tree, which may result in a long response time.

Toward this end, this paper proposes an efficient hardware implementation of a parallel priority queue.
The contributions of this paper are summarized as follows:
\begin{itemize}
\item {\bf Hole minimization:} The proposed approach minimizes holes in a parallel priority. The hole minimization techinque reduces hardware cost by XX\% and an average response time by XX\%.
\item {\bf Hardware sharing:} The hardware cost is reduced by sharing hardware between two consecutive pipelined levels. The hardware sharing technique contributes XX\% reduction in the hardware cost.
\item {\bf Replacement operation:} When a {\it min-delete} operation immediately followed by an {\it insertion} operation is detected, a {\it replacement} operation substitutes these two operations. In this way, we reduce an average response time by XX\%.
\end{itemize}
